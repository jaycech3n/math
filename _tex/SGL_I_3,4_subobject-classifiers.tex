\documentclass[a4paper, 11pt]{article}

\usepackage[top=2.5cm, bottom=3cm]{geometry}

\usepackage{amsmath, amssymb, mathtools}

\newcommand{\id}{\mathrm{id}}
\newcommand{\unit}{\mathbf{1}}
\newcommand{\Hom}{\mathrm{Hom}}
\newcommand{\Set}{\mathrm{Set}}
\newcommand{\Mono}{\mathrm{Mono}}
\newcommand{\Sub}{\mathrm{Sub}}

\usepackage{amsthm}

\newtheorem{theorem}{Theorem}
\newtheorem{lemma}[theorem]{Lemma}
\newtheorem{prop}[theorem]{Proposition}
\newtheorem{corollary}{Corollary}[theorem]

\theoremstyle{definition}
\newtheorem{defn}[theorem]{Definition}

\theoremstyle{remark}
\newtheorem{remark}[theorem]{Remark}
\newtheorem{eg}[theorem]{Example}
\newtheorem{exercise}[theorem]{Exercise}

\usepackage{tikz-cd}

\newsavebox{\pullback}
\sbox\pullback{%
\begin{tikzpicture}%
\draw (0,0) -- (1ex,0ex);%
\draw (1ex,0ex) -- (1ex,1ex);%
\end{tikzpicture}}

\title{
    Subobject Classifiers \\[1ex]
    \normalsize Following \\[0.5ex]
    \emph{Sheaves in Geometry and Logic} \\
    Chapter I, Sections 3 \& 4
}
\author{\normalsize Guide: Josh Chen}
\date{\normalsize 17 June '22 \\ Notts-Bergen SGL reading group meeting}

\begin{document}

\maketitle

We begin with the ur-example.
Recall that in $\Set$, for any $A \subseteq S$ we have
\[
\begin{tikzcd}
    A \ar[dr, phantom,
          "\usebox\pullback",
          very near start]
      \ar[d, hook, swap, "i"]
      \ar[r, "!"]             & * \ar[d, hook, "0"] \\
    S \ar[r, swap, "\chi_A"]  & \{0,1\}
\end{tikzcd}
\]
where
\[
    \chi_A(s) = \bigg\{
    \begin{aligned}
        & 0 \ \text{if}\ s \in A \\
        & 1 \ \text{otherwise.}
    \end{aligned}
\]
That is, the mono $A \xhookrightarrow{i} S$ is a pullback of the mono $* \xhookrightarrow{0} 2$ along the characteristic function $\chi_A$.
From this we can already extract the ideas for the following two definitions:

\begin{defn}
    A \emph{subobject} of $X$ is an isomorphism class of monos into $X$.
\end{defn}

To elaborate: for any category $\mathcal C$ and $X \in \mathcal C$, the monos into $X$ form a full subcategory $\Mono_{\mathcal C}(X)$ of $\mathcal C/X$.
This is a preorder, so ``being isomorphic'' is a property (not structure).
Such an isomorphism class is a subobject.

\begin{eg}
    In $\Set$, $\{0,1\} \subset \{0,1,2\}$, and the inclusion $i \colon \{0,1\} \hookrightarrow \{0,1,2\}$ witnesses $\{0,1\}$ as a subobject of $\{0,1,2\}$.
    But also the injection
    \[ f \colon \{2,3\} \rightarrowtail \{0,1,2\}, \]
    where $f(2) = 1$, $f(3) = 0$, represents (``is'') the same subobject.
\end{eg}

\begin{defn}
    A \emph{subobject classifier} in a category $\mathcal C$ with terminal $\unit$ consists of an object $\Omega$ and a mono $\top \colon \unit \rightarrow \Omega$,
    such that any mono arises as the pullback of $\top$ along a unique arrow into $\Omega$.
\end{defn}

Explicitly: for any $X \in \mathcal C$ and mono $m \colon S \rightarrowtail X$, there's a unique \emph{characteristic} or \emph{classifying map} $\chi_S \colon X \rightarrow \Omega$ making
\[
\begin{tikzcd}
    S \ar[dr, phantom,
          "\usebox\pullback",
          very near start]
      \ar[d, tail, swap, "m"]
      \ar[r, "!"]             & \unit \ar[d, tail, "\top"] \\
    X \ar[r, dashed, swap, "\chi_S"]  & \Omega
\end{tikzcd}
\]
a pullback.

\begin{exercise} \label{ex:subobj-char-map}
    If $T \rightarrowtail X$ represents the same subobject as $S \rightarrowtail X$ then
    \[
    \begin{tikzcd}
        T \ar[dr, phantom,
            "\usebox\pullback",
            very near start]
        \ar[d, tail]
        \ar[r, "!"]             & \unit \ar[d, tail, "\top"] \\
        X \ar[r, dashed, swap, "\chi_S"]  & \Omega
    \end{tikzcd}
    \]
    is also a pullback square.
    That is, the characteristic map $\chi$ is the same for isomorphic monos.
    This justifies the name ``subobject'' classifier (as opposed to ``mono'' classifier).
\end{exercise}

For every preorder $\lesssim$ we get a partial order by quotienting by
\[ \sim \ \coloneqq \ (\lesssim \cap \gtrsim). \]
Applying this to $\Mono_{\mathcal C}(X)$ gives $\Sub_{\mathcal C}(X)$, the poset of subobjects of $X$.
Forgetting the poset structure, for every object $X$ we have a class $\Sub_{\mathcal C}(X)$ of subobjects of $X$.
This is functorial: for $X \xrightarrow{f} Y$ we have the pullback map
\[ f^* \colon \mathcal C/Y \rightarrow \mathcal C/X \]
which restricts to
\[ f^* \colon \Mono_{\mathcal C}(Y) \rightarrow \Mono_{\mathcal C}(X) \]
(because pullbacks of monos are mono), and which further passes to the quotient
\[ f^* \colon \Sub_{\mathcal C}(Y) \rightarrow \Sub_{\mathcal C}(X), \]
acting via $f^*[g] = [f^*g]$ (one can check that this is well defined).
The upshot is that
\[ \Sub_{\mathcal C} \colon \mathcal C^{op} \rightarrow \Set \]
is a (possibly large) presheaf on $\mathcal C$.

\begin{defn}
    $\mathcal C$ is \emph{well powered} if for all $X \in \mathcal C$, $\Sub_{\mathcal C}(X)$ is small.
\end{defn}

What's the significance of all this to subobject classifiers?
Well, the definition of a subobject classifier as
\begin{quote}
    ``A mono $\unit \xrightarrow{\top} \Omega$ (representing a subobject), such that every subobject (represented by) $S \xrightarrow{m} X$ arises uniquely as a pullback of $\top$''
\end{quote}
amounts to saying that there is a bijection
\[ \Hom(X , \Omega) \cong \Sub_{\mathcal C}(X). \]
Furthermore, by properties of pullback, this isomorphism ends up being natural in $X$.
That is, the (target of the) subobject classifier $\Omega$ is a representing object for the subobject functor.
In fact, the converse is also true.

\begin{prop}[Proposition 1, SGL I\S3]
    A locally small category $\mathcal C$ with terminal object and ``enough'' pullbacks has a subobject classifier if and only if
    \[ \Sub_{\mathcal C} \colon \mathcal C^{op} \rightarrow \Set \]
    is representable.
\end{prop}

\begin{proof}
\noindent ($\Rightarrow$)
Let $\unit \xrightarrow{\top} \Omega$ be a subobject classifier for $\mathcal C$.
We claim that $\Omega$ is a representing object for $\Sub_{\mathcal C}$.
Firstly, for any $X \in \mathcal C$ the map
\begin{align*}
    \varphi \colon \Hom(X , \Omega) & \rightarrow \Sub_{\mathcal C}(X) \\
    f & \mapsto [f^*\top]
\end{align*}
that sends $f$ to the subobject represented by $f^*\top$ is a bijection, since, by the universal property of $\Omega$, $f$ is the unique preimage of $[f^*\top]$ under $\varphi$.

One also easily checks that $\varphi$ is also natural: for any $g \colon X \rightarrow Y$ in $\mathcal C$, the commutativity of
\[
\begin{tikzcd}
    \Hom(Y, \Omega) \ar[d, swap, "-\circ g"]
                    \ar[r, "\varphi"]        & \Sub_{\mathcal C}(Y) \ar[d, "g^*"] \\
    \Hom(X, \Omega) \ar[r, swap, "\varphi"]  & \Sub_{\mathcal C}(X)
\end{tikzcd}
\]
boils down to the fact that for any $k$, $g^*(f^*k)$ is isomorphic to $(f \circ g)^*k$ in $\mathcal C/X$.

($\Leftarrow$)
Conversely, assume that $\Omega$ is a representing object for $\Sub_{\mathcal C}$ via a natural isomorphism
\[ \psi \colon \Sub_{\mathcal C} \xrightarrow{\cong} \Hom(-,\Omega). \]
Define our candidate ``true'' subobject, represented by some mono
\[ \top \colon \Omega^\prime \rightarrowtail \Omega, \]
to be the correspondent of $\id_\Omega$ under $\psi$, i.e.\ $[\top] \coloneqq \psi^{-1}(\id_\Omega)$.

Now let $[m] \in \Sub_{\mathcal C}(X)$ be a subobject of $X$.
Take its candidate characteristic map $\chi \in \Hom(X, \Omega)$ to be its correspondent under $\psi$, i.e.\
\[ \chi \coloneqq \psi([m]). \]
Naturality of $\psi$ means that
\[ \id_\Omega \circ \chi = \psi([\top]) \circ \chi = \psi([\chi^*\top]), \]
so that $m$ and $\chi^*\top$ represent the same subobject and are thus isomorphic, and so $\chi$ is an arrow along which $m$ is a pullback.
Furthermore $\chi$ is the unique such arrow since $\psi$ is an isomorphism.

Finally it remains to show that $\Omega^\prime$ is terminal.
One easily checks that for any $X \in \mathcal C$ and $f, g \colon X \rightarrow \Omega^\prime$, the two squares
\[
\begin{tikzcd}
    X \ar[d, equal] \ar[r, "f"] & \Omega^\prime \ar[d, tail, "\top"] \\
    X \ar[r, swap, "\top \circ f"] & \Omega
\end{tikzcd}
\qquad
\begin{tikzcd}
    X \ar[d, equal] \ar[r, "g"] & \Omega^\prime \ar[d, tail, "\top"] \\
    X \ar[r, swap, "\top \circ g"] & \Omega
\end{tikzcd}
\]
are pullbacks, so that $\top \circ f = \top \circ g$ by uniqueness of the bottom arrows, and hence $f = g$ by monocity of $\top$.
\end{proof}

\begin{corollary}
    A locally small category with subobject classifier is well powered.\footnote{i.e.\ each object has set-many subobjects, which I think means (later) that the power object is nice and can be defined.}
\end{corollary}

\begin{corollary}
    Subobject classifiers are unique up to isomorphism.
\end{corollary}

\newpage

\section*{Addendum}

In the meeting Elisabeth pointed out that $\Sub_\Set$ is the contravariant powerset functor, whose action on maps is to take $f \colon X \rightarrow Y$ to the map $\mathcal P(Y) \rightarrow \mathcal P(X)$ that sends $A \subseteq Y$ to the preimage $f^{-1}(A) \subseteq X$.

There is also the \emph{covariant} powerset functor that sends $f \colon X \rightarrow Y$ to the \emph{direct image} map
\begin{align*}
    \mathcal P(X) & \rightarrow \mathcal P(Y) \\
    A & \mapsto f(A).
\end{align*}
It should be possible to generalize this to arbitrary toposes using the (regular) epi-mono factorization.
We'll possibly discuss this in the next meeting.

\end{document}
 